% source - https://www.overleaf.com/latex/templates/mtecks-resume/fzgztpkgngjc

%%%%
% MTecknology's Resume
%%%%
% Author: Michael Lustfield
% License: CC-BY-4
% - https://creativecommons.org/licenses/by/4.0/legalcode.txt
%%%%

\documentclass[a4paper,10pt]{article}
%%%%%%%%%%%%%%%%%%%%%%%
%% BEGIN_FILE: mteck.sty
%% NOTE: Everything between here and END_FILE can
%% be relocated to "mteck.sty" and then included with:
%\usepackage{mteck}

% Dependencies
% NOTE: Some packages (lastpage, hyperref) require second build!
\usepackage[empty]{fullpage}
\usepackage{titlesec}
\usepackage{enumitem}
\usepackage[colorlinks=true]{hyperref}
\usepackage{fancyhdr}
\usepackage{fontawesome5}
\usepackage{multicol}
\usepackage{bookmark}
\usepackage{lastpage}
\usepackage{graphicx}

% Sans-Serif
%\usepackage[sfdefault]{FiraSans}
%\usepackage[sfdefault]{roboto}
%\usepackage[sfdefault]{noto-sans}
%\usepackage[default]{sourcesanspro}

% Serif
%\usepackage{CormorantGaramond}
\usepackage{charter}
\usepackage[backend=biber,style=authoryear,sorting=ydnt,defernumbers=true,dashed=false,maxnames=99]{biblatex}

% Add your bibliography file
\addbibresource{publications.bib}

% Colors
% Use with \color{Name}
% Can wrap entire heading with {\color{accent} [...] }
\usepackage{xcolor}
\definecolor{accentTitle}{HTML}{0e6e55}
\definecolor{accentText}{HTML}{0e6e55}
\definecolor{accentLine}{HTML}{a16f0b}

% Misc. Options
\pagestyle{fancy}
\fancyhf{}
\fancyfoot{}
\renewcommand{\headrulewidth}{0pt}
\renewcommand{\footrulewidth}{0pt}
\urlstyle{same}

% Adjust Margins
\addtolength{\oddsidemargin}{-0.7in}
\addtolength{\evensidemargin}{-0.5in}
\addtolength{\textwidth}{1.19in}
\addtolength{\topmargin}{-0.7in}
\addtolength{\textheight}{1.4in}

\setlength{\multicolsep}{-3.0pt}
\setlength{\columnsep}{-1pt}
\setlength{\tabcolsep}{0pt}
\setlength{\footskip}{3.7pt}
\raggedbottom
\raggedright

% ATS Readability
\input{glyphtounicode}
\pdfgentounicode=1

%-----------------%
% Custom Commands %
%-----------------%

% Centered title along with subtitle between HR
% Usage: \documentTitle{Name}{Details}
\newcommand{\documentTitle}[2]{
	\begin{center}
		% Create a minipage for the main content and another for the photo
		% \begin{minipage}[c]{0.7\textwidth}
		% 	\centering
			{\Huge\color{accentTitle} #1}
			\vspace{10pt}
			{\color{accentLine} \hrule}
			\vspace{2pt}
			%{\footnotesize\color{accentTitle} #2}
			\footnotesize{#2}
			\vspace{2pt}
			{\color{accentLine} \hrule}
		% \end{minipage}
		% \hfill
		% \begin{minipage}[c]{0.25\textwidth}
		% 	\centering
		% 	\includegraphics[width=2.5cm,height=2.5cm,keepaspectratio]{photoCHTan.jpg}
		% \end{minipage}
	\end{center}
}

% Create a footer with correct padding
% Usage: \documentFooter{\thepage of X}
\newcommand{\documentFooter}[1]{
	\setlength{\footskip}{10.25pt}
	\fancyfoot[C]{\footnotesize #1}
}

% Simple wrapper to set up page numbering
% Usage: \numberedPages
% WARNING: Must run pdflatex twice!
\newcommand{\numberedPages}{
	\documentFooter{\thepage/\pageref{LastPage}}
}

% Section heading with horizontal rule
% Usage: \section{Title}
\titleformat{\section}{
	\vspace{-5pt}
	\color{accentText}
	\raggedright\large\bfseries
}{}{0em}{}[\color{accentLine}\titlerule]

% Section heading with separator and content on same line
% Usage: \tinysection{Title}
\newcommand{\tinysection}[1]{
	\phantomsection
	\addcontentsline{toc}{section}{#1}
	{\large{\bfseries\color{accentText}#1} {\color{accentLine} |}}
}

% Custom subsection aligned slightly to the right of \headingBf
% Usage: \customsubsection{Title}
\newcommand{\customsubsection}[1]{
%	\vspace{6pt}
	%\hspace{30pt}\textbf{\underline{#1}}\\[-2pt]
	\hspace{20pt}\textbf{#1}\\[-2pt]
	\vspace{4pt}
}


% Indented line with arguments left/right aligned in document
% Usage: \heading{Left}{Right}
\newcommand{\heading}[2]{
	\hspace{10pt}#1\hfill#2\\
}

% Adds \textbf to \heading
\newcommand{\headingBf}[2]{
	\heading{\textbf{#1}}{\textbf{#2}}
}

% Adds \textit to \heading
\newcommand{\headingIt}[2]{
	\heading{\textit{#1}}{\textit{#2}}
}

% Template for itemized lists
% Usage: \begin{resume_list} [items] \end{resume_list}
\newenvironment{resume_list}{
	\vspace{-7pt}
	\begin{itemize}[itemsep=-2px, parsep=1pt, leftmargin=30pt]
	}{
	\end{itemize}
	%\vspace{-2pt}
}

% Formats an item to use as an itemized title
% Usage: \itemTitle{Title}
\newcommand{\itemTitle}[1]{
	\item[] \hspace{-10pt}\textit{#1}\vspace{4pt}
}

% Bullets used in itemized lists
\renewcommand\labelitemi{--}

%% END_FILE: mteck.sty
%%%%%%%%%%%%%%%%%%%%%%


%===================%%===================%%===================%%===================%%===================%
% John Doe's Resume %
%===================%%===================%%===================%%===================%%===================%

\numberedPages % NOTE: lastpage requires a second build
%\documentFooter{\thepage of 2} % Does similar without using lastpage
\begin{document}
	
	%---------%
	% Heading %
	%---------%
	
	%Disable link color
	\hypersetup{
		linkcolor=black,
		urlcolor=black,
		citecolor=black
	}
	
	\documentTitle{Cheen Hau, Tan}{
		% Web Version
		%\raisebox{-0.05\height} \faPhone\ [redacted - web copy] ~
		%\raisebox{-0.15\height} \faEnvelope\ [redacted - web copy] ~
		%%
		\href{tel:PHONE-REDACT}{
			\raisebox{-0.05\height} \faPhone\ PHONE-REDACT} ~ | ~
		\href{mailto:EMAIL-REDACT}{
			\raisebox{-0.15\height} \faEnvelope\ EMAIL-REDACT} ~ | ~
		\href{https://linkedin.com/in/cheen-hau-tan/}{
			\raisebox{-0.15\height} \faLinkedin\ linkedin.com/in/cheen-hau-tan} ~ | ~
		\href{https://github.com/chtanch}{
			\raisebox{-0.15\height} \faGithub\ github.com/chtanch}
	}
	
	%---------%
	% Summary %
	%---------%
	
	% Enable link color
	\hypersetup{
		linkcolor=black,
		urlcolor=blue!60!black,
		citecolor=black
	}
	
	\tinysection{Summary}
	Senior software engineer specializing in generative AI (LLM, RAG, agents) and computer vision (deep learning-based computer vision and video/image processing) solutions. Experience in developing solution proposals and leading development teams. Background in both software development and research. PhD from NTU Singapore.
	
	%--------%
	% Skills %
	%--------%
	
	\section{Skills}
	
	\begin{multicols}{2}
		\begin{itemize}[itemsep=-2px, parsep=1pt, leftmargin=75pt]
			% \item[\textbf{Automation}] SaltStack, Ansible, Chef, Puppet
			% \item[\textbf{Cloud}] Salt-Cloud, Linode, GCP, AWS
			\item[\textbf{Languages}] Python, Bash, C++, Latex
			\item[\textbf{OS}] Ubuntu, Windows
			% \item[\textbf{Policies}] CIS, SOC2, PCI-DSS, GDPR, ITIL
			% \item[\textbf{Testing}] Pytest, Docker, CircleCI, Jenkins, Inspec
			% \item[\textbf{Languages}] Python, Bash, C++
			\item[\textbf{Libraries}] llamaindex, OpenAI API, Huggingface transformers, OpenCV, Scikit-learn, Streamlit, Pytorch, TensorFlow, Keras, Model Context Protocol
			\item[\textbf{CI/CD}] Git, GitHub Actions
			\item[\textbf{MLOps}] MLflow, Arize Phoenix, DVC
			\item[\textbf{Deployment}] Docker, Docker Compose
			\item[\textbf{Coding assistants}] Github Copilot, Roo Code
		\end{itemize}
	\end{multicols}
	
	%------------%
	% Experience %
	%------------%
	
	\section{Experience}

	\headingBf{Intel Malaysia (Intel Flex)}{Mar 2022 - Present}
	\headingIt{Senior AI Software Development Engineer}{}
	
	\customsubsection{Responsibilities:}
	\begin{resume_list}
		\item Contribute to software development and architecture design for AI projects.
		\item Drive innovation and learning through Proof of Concepts (POCs) and experimentation with emerging AI frameworks.
		\item Serve as TFA (Technical Focus Area) Lead for AI — plan learning roadmaps and oversee AI upskilling initiatives.
		\item Mentor and upskill junior engineers through sharing sessions, workshops, and POC-based learning.
		\item Support business development by preparing project proposals, providing technical consultation, and interviewing prospective hires.
	\end{resume_list}
	
	\customsubsection{Selected Projects/POCs:}
	\begin{resume_list}
		\itemTitle{Intel AI Assistant Builder project}{April 2024 - Present}
		\item \href{https://www.laptopmag.com/ai/mwc-2025-intel-ai-assistant-builder-for-customer-llms-and-chatbots}{Intel AI Assistant Builder} is a customizable chatbot optimized for Intel hardware and showcased in multiple exhibitions such as MWC, CES, Computex.
		\item Led Flex Malaysia team, and contributed to software development and architecture design for backend components, including RAG pipeline, workflows, LLM serving, gRPC API, CI, tests, and benchmarks. Implementation in python and llamaindex.
		\vspace{6pt}
		\itemTitle{Platform recommender chatbot project}{Jul 2024 - Dec 2024}
		\item Chatbot collects requirements from users and recommends the most suitable Intel platform. Implementation in python and langchain.
		\item Discussed requirements with customer and prepared architecture and proposal for project. Provided technical support for team to successfully deliver the project.
		\vspace{6pt}
		% \itemTitle{Flex knowledgebase chatbot POC}{Jan 2024 - June 2025}
		% \item Developed POC proposal for a python-based chatbot to answer questions on Flex's knowledgebase, and led team to develop and deploy chatbot
		% \vspace{6pt}
		\itemTitle{Intel IPEX-LLM project}{Oct 2023 - Mar 2024}
		\item \href{https://github.com/intel/ipex-llm}{Intel IPEX-LLM} is a python-based LLM acceleration library optimized for Intel hardware. Implementation in python.
		\item Contributed to debugging, validation, unit tests, benchmarking, documentation, and enabling IPEX-LLM for WebUI application.
		\vspace{6pt}
		\itemTitle{MLOps POC}{Jan 2023 - Dec 2023}
		\item Led team to implement MLOps tools and practices, including pipeline orchestration, data versioning, experiment tracking, and model serving, to streamline model training and deployment on an image classification task. Implementation in python.
		\vspace{6pt}
		\itemTitle{Intel Scenescape project}{April 2022 - Sept 2022}
		\item Developed radar processing pipeline for human detection and tracking on \href{https://www.intel.com/content/www/us/en/developer/tools/scenescape/overview.html}{Intel Scenescape} platform in C++ and scikit-learn.
		\vspace{6pt}
	\end{resume_list}
	
	\headingBf{Nanyang Technological University}{May 2016 - Feb 2022}
	\headingIt{Research Fellow}{}
	\begin{resume_list}
		\itemTitle{Car cabin monitoring project (Continental-NTU Corporate Lab)}{Jan 2021 - Feb 2022}
		\item Researched deep learning-based methods for monitoring persons and objects in car cabin. Patent filed.
		\item Conducted undergraduate lab session on deep learning-based object detection with mmdetection library
		\vspace{3pt}
		\itemTitle{Traffic monitoring project (ST Engineering-NTU Corporate Lab)}{April 2019 - Dec 2020}
		\item Developed Deep Learning-based vehicle detector and tracker for traffic monitoring. Implementation in Python and Tensorflow outperforms all other methods in UA-Detrac 2017 benchmark.
		\item Filed invention disclosure for vehicle detector and tracker. Method to be deployed for automated highway vehicle monitoring and airport traffic monitoring.
		\item Applied fine tuning of vehicle detector and tracker for local traffic. Set up and managed video annotation system.
		\item Developed lane marker detection and lane detection algorithm from traffic camera feed. Implementation in Python and OpenCV. Filed invention disclosure.
		\item Mentored a MSc student and an undergraduate research student in vehicle tracking and traffic data analysis.
		\vspace{3pt}
		\itemTitle{Rain removal project (ST Engineering-NTU Corporate Lab)}{May 2016 - March 2019}
		\item Implemented superpixel segmentation-based rain removal using C++/OpenCV and CUDA. Applied algorithm level and code level optimization to achieve near real-time operation. Filed invention disclosure for method.
		\item Published rain removal method in IEEE Conference on Computer Vision and Pattern Recognition (CVPR) 2018
		\item Implemented video-based haze removal using C++/OpenCV
		\item Publications in topics of haze/fog removal and light field camera-based rain removal
		\item Mentored two undergraduate final year students in projects involving motion detection and video stabilization
	\end{resume_list}

	\headingBf{Nanyang Technological University}{Feb 2014 - April 2016}
	\headingIt{Research Engineer}{}
	\begin{resume_list}
		\itemTitle{Rain removal project}{May 2016 - March 2019}
		\item Implemented a rain removal program using C++/OpenCV for vehicle navigation and a GUI for video capture using Pointgrey surround camera
	\end{resume_list}
	
	%-----------%
	% Education %
	%-----------%
	
	\section{Education}
	
	\headingBf{Nanyang Technological University}{Jan 2008 - Jan 2015} % Note: Adding year(s) exposes an implied age
	\headingIt{Ph.D in Electrical and Electronics Engineering}{}
	\headingIt{Awarded NTU PhD Research Scholarship.}{}
	\headingIt{Research in computer graphics mesh simplification using image-based mesh simplification methods}{}
	\headingIt{Research in motion capture data recovery using low rank estimation methods}{}

	\vspace{5pt}

	\headingBf{Nanyang Technological University}{Jul 2003 - May 2007} % Note: Adding year(s) exposes an implied age
	\headingIt{B.Eng. in Electrical and Electronics Engineering (1st class honors)}{}
	\headingIt{Awarded ASEAN Undergraduate Scholarship.}{}
	% \headingIt{Graduated under Dean's List (top 5\% of cohort).}{}
	\headingIt{Improved motion compensator block of a H.264 video decoder using VHDL for Final Year Project.}{}
	
	%--------------%
	% Publications %
	%--------------%
	
	\section{Publications}
	\nocite{*}
	
	% Use biblatex to print bibliography
	% Sort by year (descending), then by name, then by title
	\begingroup
	\footnotesize
	\setlength\bibitemsep{3pt} % Adjust spacing between items
	\printbibliography[heading=none]
	\endgroup
	
\end{document}